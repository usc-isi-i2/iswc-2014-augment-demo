We see comparisons between this approach and those used in relational database integration and semantic service composition.  
In ORCHESTRA\cite{ives2008orchestra}, much care is taken to align tables to a schema graph.  
For integration, heuristics are then applied to translate keyword searches over the graph into join paths using its Q query system, but these joins are not guaranteed to be semantically meaningful, as also demonstrated in our collaboration, CopyCat.  
By using the R2RML mappings, Karma can propose integration paths with inherent meaning.  

Meanwhile, platforms like iServe\cite{pedrinaci2010iserve} aim to capture Linked Services and make them discoverable and queryable by annotating them with their Minimal Service Model using one of two editors.
We have also repurposed R2RML as a way of capturing service descriptions in Karma as well\cite{taheriyan12:lapis}, which means we enable our users to also compose such services with other data sources like JSON, XML, and relational databases. 