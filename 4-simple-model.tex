We begin by using Karma's existing capability to create a source model, illustrated in Figure~\ref{fig:simple-model-screenshot}, using the CIDOC CRM ontology for the first CSV file.  
Karma can then translate this model into an R2RML mapping.
From this R2RML mapping, Karma can then compare it to the other mappings loaded in the Model Manager to discover new ways to integrate data.


\begin{figure*}
\begin{center}
\includegraphics[width=4.8in]{images/4-simple-model.png}
\vspace{-3mm}
\caption{A Karma user has created an R2RML mapping for a CSV file of a museum's artists' biographical records and attempts to discover new data to augment the records}
\vspace{-2mm}
\label{fig:simple-model-screenshot}
\end{center}
\vspace{-1.5em}
\end{figure*}
