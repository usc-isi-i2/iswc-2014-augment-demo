\documentclass[runningheads,a4paper]{llncs}

\usepackage{amssymb}
\setcounter{tocdepth}{3}
\usepackage{graphicx}
\usepackage{url}
\usepackage{color}
\usepackage{xspace}

\newcommand{\comment}[1]{\marginpar{\scriptsize\textcolor{red}{#1}}}
%\definecolor{Orange}{rgb}{1,0.5,0}
\newcommand{\todo}[1]{\textbf{\textcolor{red}{[[#1]]}}}

\newcommand{\karma}{\textsc{Karma}\xspace}
\newcommand{\eg}{e.g.,\xspace}
\newcommand{\ie}{i.e.,\xspace}

\urldef{\mailsa}\path|{knoblock, pszekely, slepicka}@isi.edu|
\urldef{\mailsb}\path|{chengyey}@usc.edu|    
\newcommand{\keywords}[1]{\par\addvspace\baselineskip
\noindent\keywordname\enspace\ignorespaces#1}

\begin{document}

\mainmatter  % start of an individual contribution

% first the title is needed
\title{A Demonstration of Linked Data Source Discovery and Integration}

% a short form should be given in case it is too long for the running head
\titlerunning{A Demonstration of Linked Data Discovery and Integration}

% the name(s) of the author(s) follow(s) next
%
\author{Jason Slepicka%
\and Chengye Yin\and Pedro Szekely\and Craig A. Knoblock}
%
\authorrunning{Slepicka et al.}
% (feature abused for this document to repeat the title also on left hand pages)

% the affiliations are given next; don't give your e-mail address
% unless you accept that it will be published
\institute{University of Southern California\\ 
Information Sciences Institute and Department of Computer Science, USA\\
\mailsa\\
\mailsb\\
\url{http://www.isi.edu}}

\maketitle


\begin{abstract}
%  1. The problem
The Linked Data cloud is an enormous repository of data, but it is difficult for users to find relevant data and to integrate this data into their own datasets. 
%  2. Why the problem is a problem (why it's an interesting problem [SPJ])
Datasets in the Link Data cloud are represented using ontologies, but lack accurate descriptions of the data they contain.
%  3. Startling sentence (what your solution achieves [SPJ])
We present an approach that leverages R2RML mappings to describe the contents of datasets and that uses these mappings to find relevant data, and to integrate it with a user's own datasets.
%  4. Implication of startling sentence (what follows from your solution [SPJ])
Our demonstration shows how users can easily create R2RML mappings for their datasets and then use these mappings to augment their datasets with data from the Linked Data cloud.
\end{abstract}

\subsubsection*{Introduction} 
As users begin to semantically model their data and publish it as RDF to the linked data cloud, they should be able to take advantage of their work and discover linked data, so they can integrate it into their own datasets.  
For museum curators like those at the Amon Carter Museum of American Art, this means they could multiply their efforts by doing something as simple as filling in gaps like missing birth dates in their own data. 
They could also use linked data to verify their own data and identify inconsistencies between it and another museum's like the Smithsonian.

With the explosive growth in linked data lately, however, figuring out what data is actually available is difficult.
Fortunately, as the community aligns on standard source modeling techniques like R2RML for generating RDF, it is possible to reason about the entities and relationships users are trying to capture in their source models, especially if users within a domain use the same ontologies and share their source models.
By users sharing source models along with their data, we will demonstrate how users of Karma can discover such data and integrate it with their own.


\subsubsection*{Datasets}
\begin{enumerate}
\item CSV file from a museum, containing 197 artists. %containing 2433 records of artworks by 
\item Linked Data created from the Smithsonian American Art Museum (SAAM) content management system, including over 40,000 artworks and 8,000 artists accessible by a SPARQL endpoint. 
In previous work~\cite{Szekely:2013vq} ~\todo{cite saam-lod github} we mapped the SAAM dataset to the CIDOC CRM ontology~\cite{Doerr:2003:CCR:958671.958678} using \rtworml.
In the demonstration, we are using the SAAM LOD as a proxy for the Linked Data cloud to illustrate the vision of a Linked Data cloud populated with \rtworml models.
\item \rtworml repository containing \rtworml mappings from the Web and made accessible by a \sparql endpoint (separate from the Linked Data endpoint).
\karma provides a component to manage the \rtworml repository (Fig~\ref{fig:model-manager-screenshot}).
\item owl:sameAs links between the datasets' artists generated using LIMES \cite{ngomo2011limes}.
\end{enumerate}
\begin{figure*}[bth]
\begin{center}
\includegraphics[width=4.8in]{images/3-model-manager.png}
\vspace{-3mm}
\caption{Screenshot of the model manager, showing the \rtworml mappings fetched from the GitHub repository where we shared the mappings for the SAAM dataset}
\vspace{-2mm}
\label{fig:model-manager-screenshot}
\end{center}
\vspace{-1.5em}
\end{figure*}

\subsubsection*{Model Management} 
As users develop models for different sources, they can publish their models in a repository to share with other users.  This repository can also serve as a collaboration point for users to subsequently build upon previously published models.  Karma users can use the interface illustrated in Figure~\ref{fig:model-manager-screenshot} to add models from external sources by providing a URL or remove models that are obsolete or otherwise no longer useful.  Figure~\ref{fig:model-manager-screenshot} also enumerates a number of models for data from the Smithsonian that were developed against csv files generated from database table samples.  When users want Karma to generate RDF from a future source, they can also apply an appropriate model from the repository.

\begin{figure*}
\begin{center}
\includegraphics[width=4.0in]{3-model-manager.png}
\vspace{-3mm}
\caption{A list of source models a user has made available to Karma}
\vspace{-2mm}
\label{fig:model-manager-screenshot}
\end{center}
\vspace{-1.5em}
\end{figure*}



\subsubsection*{Modeling a New Source} 
For the sake of this demonstration, we have created a subset of the original data provided by the Amon Carter Museum of American Art that describes the artists behind the works the museum curates.  
To create a new source model for the ACMAA dataset in Karma, we first need to model the primary entity in the dataset, the artist.  To make the RDF for the dataset linkable, we need to create URIs for the artists.  
We do this by providing a few lines of Python to define a URI template using other fields to create a new field containing a URI for each artist.  We then model the artist by adding an E21\_Person to the source model.  
This creates a new node labelled E21\_Person1 and we set the URI of the node to the new field.  
We also create a E41\_Appelation with a rdfs:label to capture the artist's name.  
The new source model is illustrated in Figure~\ref{fig:simple-model-screenshot}. 


\begin{figure*}
\begin{center}
\includegraphics[width=4.8in]{images/4-simple-model.png}
\vspace{-3mm}
\caption{A simple model for describing an artist as an E21\_Person}
\vspace{-2mm}
\label{fig:simple-model-screenshot}
\end{center}
\vspace{-1.5em}
\end{figure*}




To facilitate the discovery and integration, we have also generated a set of owl:sameAs links between the artists in the new dataset and the artists from the Smithsonian dataset using LIMES \cite{ngomo2011limes}.


\subsubsection*{Discovering Data Sources} 
The user then clicks on  E21\_Person1 in the \rtworml mapping and selects Augment Data to discover new data to integrate into artist records.  
\karma retrieves \rtworml mappings in its repository that describe crm:E21\_Person, and uses these mappings to generate a candidate set of linked data sources to integrate, identifies meaningful object and data properties, and presents them to the user as illustrated in Figure~\ref{fig:search-screenshot}.
To help the users select properties to integrate, Karma uses Bloom filters to estimate the number of artists that have each of the properties listed in Figure~\ref{fig:search-screenshot}.
\begin{figure*}[t]
\centering
\includegraphics[width=4.9in]{images/5-search.png}
\vspace{-5mm}
\caption{A Karma user selects CIDOC CRM object and data properties discovered from other sources to augment crm:E21\_Person}
\vspace{-15pt}
\label{fig:search-screenshot}
\end{figure*}

\subsubsection*{Integrating Data Sources} 
The user selects the artist's biography (for completeness) and birth (for validation). Karma automatically constructs SPARQL queries for the data.
To facilitate integration, we generated owl:sameAs links between the artists using LIMES \cite{ngomo2011limes}.  
We plan to support the linking in Karma.
Karma then merges the results and updates the mapping shown in Figure~\ref{fig:augment-screenshot}, so the user can continue to explore the integrated data interactively. 
\begin{figure*}[bt]
\centering
\includegraphics[width=4.9in]{images/6-augment.png}
\vspace{-5mm}
\caption{A Karma user has integrated biographical data from the Smithsonian as new columns in their dataset. The columns contain artists' biographies and birth dates.}
\vspace{-15pt}
\label{fig:augment-screenshot}
\end{figure*}

\bibliographystyle{plain}
\bibliography{refs}

\end{document}
